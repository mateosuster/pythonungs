%\documentclass[a4paper,12pt]{article}
\documentclass[a4paper,11pt,landscape,twocolumn]{article}
%\documentclass[a4paper]{memoir}
\usepackage[utf8]{inputenc}
\usepackage[spanish, es-tabla, es-nodecimaldot]{babel}
\usepackage{amsmath}  %permite usar \text{} en el entorno Matemática
\usepackage{amssymb} % para el de números reales
%\usepackage{fancyhdr} %para encabezados y pies de página lindos
%\usepackage{lastpage} %para poder referenciar el número de la última página
\usepackage{graphicx} %para insertar gráficos
\usepackage{float} %para que funcione el H de la posición de las figuras
%\usepackage{chngpage} %para cambiar márgenes temporalmente. Por ejemplo tabla o figura un poco más grande que el text width
%\usepackage[format=plain, indention=0cm, font=small, labelfont=bf, labelsep=period, textfont=sl]{caption} %tuneado del caption de las figuras
\usepackage{mathtools} % para usar dcases, la versión displayMath de cases (funciones partidas)
\usepackage{enumerate} %para personalizar los enumeradores
\usepackage{framed} %para poner párrafos adentro de un caja con marco
%\usepackage{hyperref} %para que el índice tenga enlaces internos
%\usepackage{lastpage} %para poder referenciar el número de la última página
%\usepackage{fullpage}
\usepackage[cm]{fullpage}
\usepackage{wrapfig} %para poner tablas o figuras con texto alrededor.
\usepackage{array}
\usepackage{hyperref}
\usepackage{epigraph}

\newcolumntype{x}[1]{>{\centering\arraybackslash\hspace{0pt}}p{#1}}

%\setlength{\columnseprule}{0.5pt}
\setlength{\columnsep}{1cm}

\setlength{\epigraphwidth}{0.36\textwidth}
%\setlength{\epigraphwidth}{0.7\textwidth}


%\title{Taller de Problemas Com. 4 - Matemática FFyB - 2do. cuat. 2018}
%\author{G. Sebastián Pedersen\\ sebasped@gmail.com}
%\date{Agosto de 2018}


\renewcommand{\arraystretch}{1.3}  %para que las celdas de las tablas sean un poco más altas y entre bien el Q moño.


\begin{document}
\epigraph{La educación no es llenar una cubeta, sino encender un fuego.}{William Butler Yeats (1865-1939)}	
	%\maketitle
\noindent Autor: G. Sebastián Pedersen --- sebasped@gmail.com --- Sáb 04-May-2019.
%\begin{framed}
%	\centering
%	\noindent Los cálculos deben ir acompañados de explicaciones que aclaren su significado. Un resultado suelto, no acompañado de explicación se considerará como problema no resuelto.
%\end{framed}
\subsection*{Ejercicios} 
\begin{enumerate}
	\item Explicá qué hace la siguiente función \verb|lossimpsons|:
	\begin{verbatim}
	def lossimpsons(x):
	    devolver = "babosos"
	    
	    if x == 2 :
	        devolver = "te vas al demonio Krabappel"    
	    
	    return devolver
	\end{verbatim}

\item Explicá qué hace el siguiente programa:
\begin{verbatim}
lista = [2,15,7]
suma = 0

valor_a_sumar = lista[0]
suma = suma + valor_a_sumar

valor_a_sumar = lista[1]
suma = suma + valor_a_sumar

valor_a_sumar = lista[2]
suma = suma + valor_a_sumar

print(suma)
\end{verbatim}
\newpage
	\item Explicá qué hace el siguiente programa:
	\begin{verbatim}
	lista = [2,15,7]
	suma = 0
	
	longitud = len(lista)
	elnombrequesemecanta = 0
	
	while elnombrequesemecanta < longitud :
	    valor_a_sumar = lista[elnombrequesemecanta]
	    suma = suma + valor_a_sumar
	    elnombrequesemecanta = elnombrequesemecanta + 1
	
	print(suma)
	\end{verbatim}


\item Explicá qué hace la siguiente función \verb|laloLanda|:
\begin{verbatim}
def laloLanda(lista):

    long = len(lista)
    devolver = "Lisa, la reina de la ciencias"
    nelson = 0

    while nelson < long :
        if lista[nelson] == 10 :
            devolver = "En tu cara Flanders"
        nelson = nelson + 1

    return devolver
\end{verbatim}

\item Escribí una función que reciba como entrada una lista de números, y que devuelva como salida el promedio de todos los números de la lista.
\item Escribí una función que reciba como entrada una lista de números, y que devuelva como salida el promedio de todos los números de la lista \emph{si la longitud de la lista es menor a 5}, y que devuelva ``Ouch!" si la longitud de la lista es 5 o más.
\end{enumerate}


\end{document}