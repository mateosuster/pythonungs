\documentclass[9pt]{beamer}
%\documentclass[handout,9pt]{beamer}

\mode<presentation>
{
%  \usetheme{Warsaw}
  % or ...

%  \setbeamercovered{transparent}
  % or whatever (possibly just delete it)
}


\setbeamertemplate{navigation symbols}{}

\usepackage[utf8]{inputenc}
\usepackage[spanish, es-tabla, es-nodecimaldot]{babel}

\usepackage{listings}

%\usepackage{times}
%\usepackage[T1]{fontenc}
% Or whatever. Note that the encoding and the font should match. If T1
% does not look nice, try deleting the line with the fontenc.


\title[Intro Prog] % (optional, use only with long paper titles)
{Introducción a la programación}

\subtitle
{Usando Python}

\author[SP]
{G. Sebastián Pedersen}%\inst{1}}
% - Give the names in the same order as the appear in the paper.
% - Use the \inst{?} command only if the authors have different
%   affiliation.

\institute[UNGS] % (optional, but mostly needed)
{
%  \inst{1}%
  Instituto de Industria\\
  Universidad Nacional de General Sarmiento
%  \and
%  \inst{2}%
%  Department of Theoretical Philosophy\\
%  University of Elsewhere}
% - Use the \inst command only if there are several affiliations.
% - Keep it simple, no one is interested in your street address.
}

\date[] % (optional, should be abbreviation of conference name)
{Matemática para Economistas III, 1er. cuat. 2019\\ 
\vspace{.5cm}
\url{https://sebasped.github.io/pythonungs/}
}



% If you have a file called "university-logo-filename.xxx", where xxx
% is a graphic format that can be processed by latex or pdflatex,
% resp., then you can add a logo as follows:

% \pgfdeclareimage[height=0.5cm]{university-logo}{university-logo-filename}
% \logo{\pgfuseimage{university-logo}}



% Delete this, if you do not want the table of contents to pop up at
% the beginning of each subsection:
%\AtBeginSubsection[]
%{
%  \begin{frame}<beamer>{Outline}
%    \tableofcontents[currentsection,currentsubsection]
%  \end{frame}
%}


% If you wish to uncover everything in a step-wise fashion, uncomment
% the following command: 

%\beamerdefaultoverlayspecification{<+->}


\begin{document}

\begin{frame}
  \titlepage
\end{frame}


\begin{frame}[fragile]{Repaso clase anterior}
%\begin{block}{Programa}
¿Qué hace el siguiente programa en Python?
\footnotesize{\begin{verbatim}
# m2k_v1.py
# Ingreso las millas (en float para no hacer división entera)
mills = 34.122

# Hago la conversión a kilómetros
kms = 34.122*1.6

# Imprimo por pantalla el resultado
print mills,"millas son",kms,"kilómetros"
\end{verbatim}}\pause
%\end{block}\pause
La salida queda así: 34.122 millas son 54.5952 kilómetros\pause
\begin{itemize}
%	\item ¿Qué hace el anterior programa?\pause
	\item ¿Qué problemas tiene?\pause 
		\begin{itemize}
			\item \alert{Duplicación de información:} 34.122 lo estoy guardando en mills pero después vuelvo a poner 34.112 en kms
		\end{itemize}\pause
	\item ¿Cómo se podría arreglar?
\end{itemize}

\end{frame}



\begin{frame}[fragile]{Repaso clase anterior}
%\begin{block}{Programa}
Lo mismo \alert{sin duplicar información}
\footnotesize{\begin{verbatim}
# m2k_v2.py
	# Ingreso las millas (en float para no hacer división entera)
	mills = 34.122
	
	# Hago la conversión a kilómetros
	kms = mills*1.6
	
	# Imprimo por pantalla el resultado
	print mills,"millas son",kms,"kilómetros"
	\end{verbatim}}\pause
%\end{block}\pause
La salida sigue igual: 34.122 millas son 54.5952 kilómetros\pause
\begin{itemize}
	%	\item ¿Qué hace el anterior programa?\pause
	\item \alert{Duplicación de información corregida:} 34.122 lo estoy guardando en mills, y después uso la variable mills y no vuelvo a poner 34.112 en kms
	
\end{itemize}

\end{frame}


\begin{frame}[fragile]{Repaso clase anterior}
%\begin{block}{Programa}
¿Qué hace el siguiente programa en Python?
\footnotesize{\begin{verbatim}
# m2k_v3.py
# Ingreso las millas
mills = 34.122  # la ingreso float para no hacer división entera

# Hago la conversión a kilómetros
kms = mills*1.6

round(54.592, 2)

# Imprimo por pantalla el resultado
print mills,"millas son",kms,"kilómetros"
	\end{verbatim}}\pause
%\end{block}\pause
La salida sigue igual: 34.122 millas son 54.5952 kilómetros\pause
\begin{itemize}
	%	\item ¿Qué hace el anterior programa?\pause
	\item Lo mismo que la versión 2 (\verb|m2k_v2.py|) \pause
	\item Bah, casi... ¿Qué pasa con \verb|round(54.592, 2)|? \pause 
	\begin{itemize}
		\item \verb|round(54.592, 2)| redondea 54.592 a dos decimales después de la coma, pero \alert{el resultado no se guarda en ningún lado} ¿por qué?
	\end{itemize}\pause
	\item ¿Cómo se podría guardar el resultado del redondeo para utilizarlo después en el programa?\pause
	\item ¿Qué otro problema hay?\pause\hspace{.1cm}Nuevamente \alert{duplicación de información}: en vez de 54.592 podemos usar \verb|kms| que es la variable que guarda ese valor.
\end{itemize}

\end{frame}



\begin{frame}[fragile]{Repaso clase anterior}
%\begin{block}{Programa}
Guardo el resultado de redondear, y reutilizo el valor de \verb|kms|
\footnotesize{\begin{verbatim}
	# m2k_v4.py
	# Ingreso las millas
	mills = 34.122  # la ingreso float para no hacer división entera
	
	# Hago la conversión a kilómetros
	kms = mills*1.6
	
	kmsRedon = round(kms, 2)
	
	# Imprimo por pantalla el resultado
	print mills,"millas son",kms,"kilómetros"
	\end{verbatim}}\pause
%\end{block}\pause
La salida sigue igual: 34.122 millas son 54.5952 kilómetros\pause
\begin{itemize}
		\item ¿Qué hace ahora el programa?\pause
		\begin{itemize}
			\item Esencialmente, todavía lo mismo que la versión 2 (\verb|m2k_v2.py|)...
		\end{itemize}\pause
		\item ¿Uso en algún lado el valor de \verb|kmsRedon|?\pause
		\begin{itemize}
			\item No... ¿cómo lo podría usar?\pause
			\item En vez de mostrar (imprimir por pantalla) el valor de \verb|kms|, podría mostrar el valor redondeado \verb|kmsRedon|
		\end{itemize}
\end{itemize}

\end{frame}



\begin{frame}[fragile]{Repaso clase anterior}
%\begin{block}{Programa}
Mostrando el valor redondeado de los kilómetros
\footnotesize{\begin{verbatim}
	# m2k_v5.py
	# Ingreso las millas
	mills = 34.122  # la ingreso float para no hacer división entera
	
	# Hago la conversión a kilómetros
	kms = mills*1.6
	
	kmsRedon = round(kms, 2)
	
	# Imprimo por pantalla el resultado
	print mills,"millas son",kmsRedon,"kilómetros"
	\end{verbatim}}\pause

La salida ahora son kms redondeados: 34.122 millas son 54.6 kilómetros%\pause
\end{frame}



\begin{frame}[fragile]{Utilizando listas}
¿Qué hace el siguiente programa?
\footnotesize{\begin{verbatim}
# m2k_v6.py
# Ingreso las millas
mills = [34.122, 17.588, 3.187]  # la ingreso float para no hacer división entera

# Hago la conversión a kilómetros
kms = [mills[0]*1.6, mills[1]*1.6, mills[2]*1.6] 

# Imprimo por pantalla el resultado
print mills[0],"millas son",kms[0],"kilómetros"
print mills[1],"millas son",kms[1],"kilómetros"
print mills[2],"millas son",kms[2],"kilómetros"
	\end{verbatim}}\pause
La salida ahora queda:\\

34.122 millas son 54.5952 kilómetros\\
17.588 millas son 28.1408 kilómetros\\
3.187 millas son 5.0992 kilómetros\\ \pause

\begin{itemize}
	\item ¿Cuál es la diferencia fundamental con los anteriores?\pause
		\begin{itemize}
			\item Esencialmente que ahora son listas de floats y no un único float.
		\end{itemize}\pause	
	\item ¿Tiene información duplicada?\pause
	\begin{itemize}
		\item El \verb|1.6| aparece 3 veces... ¿podríamos evitar esta duplicación de información?\pause
		\item Sí, guardar el \verb|1.6| en una variable, y utilizarla en vez del \verb|1.6|
	\end{itemize}
\end{itemize}

\end{frame}



\begin{frame}[fragile]{Graficando el resultado}
\footnotesize{\begin{verbatim}
	# m2k_v7.py
	
	# importo la librería para graficar
	import matplotlib.pyplot as plt
	
	# Ingreso las millas
	mills = [34.122, 17.588, 3.187]  # la ingreso float para no hacer división entera
	
	# Hago la conversión a kilómetros
	kms = [mills[0]*1.6, mills[1]*1.6, mills[2]*1.6] 
	
	# Imprimo por pantalla el resultado
	print mills[0],"millas son",kms[0],"kilómetros"
	print mills[1],"millas son",kms[1],"kilómetros"
	print mills[2],"millas son",kms[2],"kilómetros"
	
	# Imprimo el resultado en un gráfico
	plt.plot(mills,kms,'*')  # plot genera un gráfico X contra Y
	plt.show()  # show muestra el gráfico
	\end{verbatim}}\pause
\begin{columns}
	\begin{column}{0.5\textwidth}
		La salida queda así:\\
		\vspace{.5cm}
		\centering
		34.122 millas son 54.5952 kilómetros\\
		17.588 millas son 28.1408 kilómetros\\
		3.187 millas son 5.0992 kilómetros
	\end{column}
	\begin{column}{0.5\textwidth}  %%<--- here
		\begin{center}
			\includegraphics[height=2.5cm]{m2kpy_graf1.png}
		\end{center}
	\end{column}
\end{columns}
\end{frame}



\begin{frame}[fragile]{Que funcione para lista de cualquier longitud}
¿Qué hace el siguiente programa?
\footnotesize{\begin{verbatim}
# m2k_v8.py
import matplotlib.pyplot as plt

# Ingreso las millas
mills = [1.1, 33.4, 13.2, 60.0, 17.35]  # la ingreso float para no hacer división entera

# guardo la longitud de la lista
long = len(mills)

# Hago la conversión a kilómetros
kms = [0]*long  # lista de ceros de longitud adecuada

i = 0 # variable para posición en la lista
while i < long : # mientras que la posición no se salga de la lista
    kms[i] = mills[i]*1.6  # convertir
    i = i + 1  # avanzar una posición

# Imprimo el resultado en un gráfico
plt.plot(mills,kms,'*')
plt.show()
\end{verbatim}}\pause
\begin{columns}
	\begin{column}{0.5\textwidth}
		La salida queda así:\\
\begin{center}
	\includegraphics[height=2cm]{m2kpy_graf2.png}\pause
\end{center}
	\end{column}
	\begin{column}{0.5\textwidth}  %%<--- here
		\begin{itemize}
			\item ¿Funciona este programa para lista de cualquier longitud?\pause
			\item Sí, porque me fijo la longitud de la lista y la recorro toda.
		\end{itemize}
	\end{column}
\end{columns}

\end{frame}




\begin{frame}{¡A codear se ha dicho!}
	Tener presente que:
	\begin{itemize}
		\item Cualquier \emph{valor que quiera ser reutilizado} (para un cálculo posterior, para una salida, etc.) \alert{debe ser almacenado previamente en una variable}
		\item Siempre que sea posible \alert{no duplicar información} en el código.
	\end{itemize}
\begin{block}{Ejercicios:}
	\begin{enumerate}
		\item Escribir un script en un archivo d2p.py. El programa tiene que tomar como entrada una lista de dólares, y convertirla a una lista de sus equivalentes en pesos (tomar algún tipo de cambio).
		\begin{itemize}
			\item El programa tiene que funcionar para lista de cualquier longitud
			\item La salida del programa tiene que ser un gráfico Dólares vs. Pesos.
		\end{itemize}
		\item Escribir un script en un archivo variacion.py. El programa tiene que tomar como entrada una lista de cotizaciones (en pesos) del dólar, y convertirla a una lista de variaciones porcentuales en la cotización.
		\begin{itemize}
			\item El programa tiene que funcionar para lista de cualquier longitud.
			\item La salida del programa tiene que ser un gráfico de la evolución de las variaciones.
		\end{itemize}
		\item Modificar el script del punto 1. y guardarlo en un archivo d2p\_v2.py. Ahora la salida del programa además del gráfico, también debe incluir una salida por pantalla indicando para cada monto de dólares su correspondiente monto equivalente en pesos. 
	\end{enumerate}
	%	
\end{block}

\end{frame}






%%
%%% All of the following is optional and typically not needed. 
%%\appendix
%%\section<presentation>*{\appendixname}
%%\subsection<presentation>*{For Further Reading}
%%
%%\begin{frame}[allowframebreaks]
%%  \frametitle<presentation>{For Further Reading}
%%    
%%  \begin{thebibliography}{10}
%%    
%%  \beamertemplatebookbibitems
%%  % Start with overview books.
%%
%%  \bibitem{Author1990}
%%    A.~Author.
%%    \newblock {\em Handbook of Everything}.
%%    \newblock Some Press, 1990.
%% 
%%    
%%  \beamertemplatearticlebibitems
%%  % Followed by interesting articles. Keep the list short. 
%%
%%  \bibitem{Someone2000}
%%    S.~Someone.
%%    \newblock On this and that.
%%    \newblock {\em Journal of This and That}, 2(1):50--100,
%%    2000.
%%  \end{thebibliography}
%%\end{frame}

\end{document}


