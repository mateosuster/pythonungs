\documentclass[9pt]{beamer}
%\documentclass[handout,9pt]{beamer}

\mode<presentation>
{
%  \usetheme{Warsaw}
  % or ...

%  \setbeamercovered{transparent}
  % or whatever (possibly just delete it)
}


\setbeamertemplate{navigation symbols}{}

\usepackage[utf8]{inputenc}
\usepackage[spanish, es-tabla, es-nodecimaldot]{babel}

\usepackage{listings}

%\usepackage{times}
%\usepackage[T1]{fontenc}
% Or whatever. Note that the encoding and the font should match. If T1
% does not look nice, try deleting the line with the fontenc.


\title[Intro Prog] % (optional, use only with long paper titles)
{Introducción a la programación}

\subtitle
{Usando Python}

\author[SP]
{G. Sebastián Pedersen}%\inst{1}}
% - Give the names in the same order as the appear in the paper.
% - Use the \inst{?} command only if the authors have different
%   affiliation.

\institute[UNGS] % (optional, but mostly needed)
{
%  \inst{1}%
  Instituto de Industria\\
  Universidad Nacional de General Sarmiento
%  \and
%  \inst{2}%
%  Department of Theoretical Philosophy\\
%  University of Elsewhere}
% - Use the \inst command only if there are several affiliations.
% - Keep it simple, no one is interested in your street address.
}

\date[] % (optional, should be abbreviation of conference name)
{Matemática para Economistas III, 1er. cuat. 2019\\ 
\vspace{.5cm}
\url{https://sebasped.github.io/pythonungs/}
}



% If you have a file called "university-logo-filename.xxx", where xxx
% is a graphic format that can be processed by latex or pdflatex,
% resp., then you can add a logo as follows:

% \pgfdeclareimage[height=0.5cm]{university-logo}{university-logo-filename}
% \logo{\pgfuseimage{university-logo}}



% Delete this, if you do not want the table of contents to pop up at
% the beginning of each subsection:
%\AtBeginSubsection[]
%{
%  \begin{frame}<beamer>{Outline}
%    \tableofcontents[currentsection,currentsubsection]
%  \end{frame}
%}


% If you wish to uncover everything in a step-wise fashion, uncomment
% the following command: 

%\beamerdefaultoverlayspecification{<+->}


\begin{document}

\begin{frame}
  \titlepage
\end{frame}



\begin{frame}[fragile]{Repaso clase anterior}
Evitar \alert{duplicar información}
\footnotesize{\begin{verbatim}
	# m2k_v2.py
	# Ingreso las millas (en float para no hacer división entera)
	mills = 34.122
	
	# Hago la conversión a kilómetros
	kms = mills*1.6
	
	# Imprimo por pantalla el resultado
	print mills,"millas son",kms,"kilómetros"
	\end{verbatim}}%\pause
\begin{itemize}
	\item \alert{Duplicación de información corregida:} 34.122 lo estoy guardando en mills, y después uso la variable mills y no vuelvo a poner 34.112 en kms
\end{itemize}
\end{frame}


\begin{frame}[fragile]{Repaso clase anterior}
La última versión usaba el ciclo \verb|while| para que el programa funcione para listas de \emph{cualquier longitud}.
\footnotesize{\begin{verbatim}
# m2k_v8.py
import matplotlib.pyplot as plt

# Ingreso las millas
mills = [1.1, 33.4, 13.2, 60.0, 17.35]  # la ingreso float para no hacer división entera

# guardo la longitud de la lista
long = len(mills)

# Hago la conversión a kilómetros
kms = [0]*long  # lista de ceros de longitud adecuada

i = 0 # variable para posición en la lista
while i < long : # mientras que la posición no se salga de la lista
    kms[i] = mills[i]*1.6  # convertir
    i = i + 1  # avanzar una posición

# Imprimo el resultado en un gráfico
plt.plot(mills,kms,'*')
plt.show()
\end{verbatim}}

\end{frame}



\begin{frame}[fragile]{Particionando el problema y encapsulando en funciones}
¿Qué hace el siguiente programa?
\footnotesize{\begin{verbatim}
# m2k_v9.py
import matplotlib.pyplot as plt

def convertir(lista):
   long = len(lista)  #guardo la longitud de la lista

    # Hago la conversión
    listaConvertida = [0]*long  # lista de ceros de longitud adecuada
    i = 0
    while i < long :
        listaConvertida[i] = lista[i]*1.6
        i = i + 1

    return listaConvertida

# El resto del código que más prolijo
mills = [1.1, 33.4, 13.2, 60.0, 17.35]

kms = convertir(mills)

plt.plot(mills,kms,'*')
plt.show()
\end{verbatim}}\pause
\begin{itemize}
	\item La función \verb|convertir| \alert{encapsula} el problema de la conversión a kms.\pause
	\item El resto del código \alert{encapsula} el problema de usar la conversión y graficar el resultado.\pause
	\item Ventajas de esto:\pause
	\begin{itemize}
		\item \footnotesize{Particionamos} el problema es dos problemas de menor complejidad.\pause
		\item Los dos problemas pueden sen resueltos por diferentes personas y al mismo tiempo.
	\end{itemize}
%	\item ¿Cómo se podría arreglar?
\end{itemize}

\end{frame}



\begin{frame}[fragile]{Particionando el problema y encapsulando en funciones}
¿Cuál es la diferencia con el anterior programa?
\footnotesize{\begin{verbatim}
# m2k_v10.py
import matplotlib.pyplot as plt
	
def convertir(lista,factor):
    long = len(lista)  #guardo la longitud de la lista
	
    # Hago la conversión
    listaConvertida = [0]*long  # lista de ceros de longitud adecuada
    i = 0
    while i < long :
        listaConvertida[i] = lista[i]*factor
        i = i + 1
    
    return listaConvertida
	
# El resto del código que más prolijo
mills = [1.1, 33.4, 13.2, 60.0, 17.35]
	
fc = 1.6
kms = convertir(mills,fc)
	
plt.plot(mills,kms,'*')
plt.show()
\end{verbatim}}\pause
\begin{itemize}
	\item La función \verb|convertir| ahora toma dos parámetros de entrada.\pause
	\item La ventaja de esto es que ahora la función \verb|convertir| funciona \alert{para cualquier lista y para cualquier factor de conversión}.
%	\item Ventajas de esto:\pause
%	\begin{itemize}
%		\item \footnotesize{Particionamos} el problema es dos problemas de menor complejidad.\pause
%		\item Los dos problemas pueden sen resueltos por diferentes personas y al mismo tiempo.
%	\end{itemize}
	%	\item ¿Cómo se podría arreglar?
\end{itemize}

\end{frame}


\begin{frame}{¡Hacking time!}
	Tener presente que:
	\begin{itemize}
		\item Cualquier \emph{valor que quiera ser reutilizado} (para un cálculo posterior, para una salida, etc.) \alert{debe ser almacenado previamente en una variable}.
		\item Siempre que sea posible \alert{no duplicar información} en el código.
		\item Siempre que sea posible \alert{particionar} el problema y \alert{encapsularlo} en problemas menos complejos mediante \alert{funciones}.
	\end{itemize}
\begin{block}{Ejercicios:}
	\begin{enumerate}
		\item Escribir un script en un archivo d2p\_v3.py. El programa tiene que tomar como entrada una lista de dólares, y convertirla a una lista de sus equivalentes en pesos (tomar algún tipo de cambio).
		\begin{itemize}
			\item El programa tiene que funcionar para listas de cualquier longitud.
			\item El programa debe utilizar una función que realice la conversión.
			\item La salida del programa tiene que ser un gráfico Dólares vs. Pesos.
		\end{itemize}
%		\item Escribir un script en un archivo variacion.py. El programa tiene que tomar como entrada una lista de cotizaciones (en pesos) del dólar, y convertirla a una lista de variaciones porcentuales en la cotización.
%		\begin{itemize}
%			\item El programa tiene que funcionar para lista de cualquier longitud.
%			\item La salida del programa tiene que ser un gráfico de la evolución de las variaciones.
%		\end{itemize}
		\item Modificar el script del punto 1. y guardarlo en un archivo d2p\_v4.py. Ahora la función que hace la conversión debe también funcionar para cualquier tipo de cambio.
	\end{enumerate}
\end{block}

\end{frame}






%%
%%% All of the following is optional and typically not needed. 
%%\appendix
%%\section<presentation>*{\appendixname}
%%\subsection<presentation>*{For Further Reading}
%%
%%\begin{frame}[allowframebreaks]
%%  \frametitle<presentation>{For Further Reading}
%%    
%%  \begin{thebibliography}{10}
%%    
%%  \beamertemplatebookbibitems
%%  % Start with overview books.
%%
%%  \bibitem{Author1990}
%%    A.~Author.
%%    \newblock {\em Handbook of Everything}.
%%    \newblock Some Press, 1990.
%% 
%%    
%%  \beamertemplatearticlebibitems
%%  % Followed by interesting articles. Keep the list short. 
%%
%%  \bibitem{Someone2000}
%%    S.~Someone.
%%    \newblock On this and that.
%%    \newblock {\em Journal of This and That}, 2(1):50--100,
%%    2000.
%%  \end{thebibliography}
%%\end{frame}

\end{document}


