\documentclass[a4paper,12pt]{article}
%\documentclass[a4paper,11pt,landscape,twocolumn]{article}
%\documentclass[a4paper]{memoir}
\usepackage[utf8]{inputenc}
\usepackage[spanish, es-tabla, es-nodecimaldot]{babel}
\usepackage{amsmath}  %permite usar \text{} en el entorno Matemática
\usepackage{amssymb} % para el de números reales
%\usepackage{fancyhdr} %para encabezados y pies de página lindos
%\usepackage{lastpage} %para poder referenciar el número de la última página
\usepackage{graphicx} %para insertar gráficos
\usepackage{float} %para que funcione el H de la posición de las figuras
%\usepackage{chngpage} %para cambiar márgenes temporalmente. Por ejemplo tabla o figura un poco más grande que el text width
%\usepackage[format=plain, indention=0cm, font=small, labelfont=bf, labelsep=period, textfont=sl]{caption} %tuneado del caption de las figuras
\usepackage{mathtools} % para usar dcases, la versión displayMath de cases (funciones partidas)
\usepackage{enumerate} %para personalizar los enumeradores
\usepackage{framed} %para poner párrafos adentro de un caja con marco
%\usepackage{hyperref} %para que el índice tenga enlaces internos
%\usepackage{lastpage} %para poder referenciar el número de la última página
%\usepackage{fullpage}
%\usepackage[cm]{fullpage}
\usepackage{wrapfig} %para poner tablas o figuras con texto alrededor.
\usepackage{array}
\usepackage{hyperref}
\usepackage{epigraph}

\newcolumntype{x}[1]{>{\centering\arraybackslash\hspace{0pt}}p{#1}}

%\setlength{\columnseprule}{0.5pt}
\setlength{\columnsep}{1cm}

\setlength{\epigraphwidth}{0.42\textwidth}
%\setlength{\epigraphwidth}{0.7\textwidth}


%\title{Taller de Problemas Com. 4 - Matemática FFyB - 2do. cuat. 2018}
%\author{G. Sebastián Pedersen\\ sebasped@gmail.com}
%\date{Agosto de 2018}


\renewcommand{\arraystretch}{1.3}  %para que las celdas de las tablas sean un poco más altas y entre bien el Q moño.


\begin{document}
\epigraph{Me lo contaron y lo olvidé; lo vi y lo entendí; lo hice y lo aprendí.}{Confucio (551-479 a.C.)}
	%\maketitle
\noindent Autor: G. Sebastián Pedersen --- sebasped@gmail.com --- Vie 03-May-2019.
%\begin{framed}
%	\centering
%	\noindent Los cálculos deben ir acompañados de explicaciones que aclaren su significado. Un resultado suelto, no acompañado de explicación se considerará como problema no resuelto.
%\end{framed}
\section*{Ejercicio} Te dan la siguiente función en Python:
\begin{verbatim}
def func(x):
    
    l = len(x)
    m = x[0]
    i=1
    
    while i < l :
        if x[i] > m :
            m = x[i]
        i=i+1
    
    return m

\end{verbatim}
\begin{enumerate}
	\item ¿De qué tipo de dato pueden ser la entrada y la salida?
	\item ¿Qué hace la función y cómo?
\end{enumerate}


\end{document}