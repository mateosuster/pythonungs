\documentclass{beamer}
%\documentclass[handout]{beamer}

\mode<presentation>
{
%  \usetheme{Warsaw}
  % or ...

%  \setbeamercovered{transparent}
  % or whatever (possibly just delete it)
}


\setbeamertemplate{navigation symbols}{}

\usepackage[utf8]{inputenc}
\usepackage[spanish, es-tabla, es-nodecimaldot]{babel}
\usepackage{xcolor}

\usepackage{listings}

%New colors defined below
\definecolor{codegreen}{rgb}{0,0.6,0}
\definecolor{codegray}{rgb}{0.5,0.5,0.5}
\definecolor{codepurple}{rgb}{0.58,0,0.82}
\definecolor{backcolour}{rgb}{0.95,0.95,0.92}

%Code listing style named "mystyle"
\lstdefinestyle{mystyle}{
  backgroundcolor=\color{backcolour},   commentstyle=\color{codegreen},
  keywordstyle=\color{magenta},
  numberstyle=\tiny\color{codegray},
  stringstyle=\color{codepurple},
  basicstyle=\ttfamily\footnotesize,
  breakatwhitespace=false,         
  breaklines=true,                 
  captionpos=b,                    
  keepspaces=true,                 
  numbers=left,                    
  numbersep=5pt,                  
  showspaces=false,                
  showstringspaces=false,
  showtabs=false,                  
  tabsize=2
}

%"mystyle" code listing set
\lstset{style=mystyle}


%\usepackage{times}
%\usepackage[T1]{fontenc}
% Or whatever. Note that the encoding and the font should match. If T1
% does not look nice, try deleting the line with the fontenc.


\title[Intro Prog] % (optional, use only with long paper titles)
{Introducción a Python}

\subtitle
{Clase 2}

\author[MS]
{Mateo Suster \\ mateosuster@gmail.com}%\inst{1}}
% - Give the names in the same order as the appear in the paper.
% - Use the \inst{?} command only if the authors have different
%   affiliation.

\institute[UNGS] % (optional, but mostly needed)
{
%  \inst{1}%
  Matemática para Economistas III \\ 
  Instituto de Industria\\
  Universidad Nacional de General Sarmiento
}

\date[] % (optional, should be abbreviation of conference name)
{ \today}



\begin{document}

\begin{frame}
  \titlepage
\end{frame}


% PAG 1
\begin{frame}[fragile]{¿Qué hace el siguiente programa en Python?} 
\begin{lstlisting}[language=Python]
#ingreso las millas 
mills = 34.122

#hago la conversion a kilometros
km = 34.122 * 1.6

#imprimo por pantalla el resultado
print(mills, "millas son", km, "kilometros")
	
#la salida queda 
> 34.122 millas son 54.595200000000006 kilometros
\end{lstlisting}  \pause

¿Qué problemas tiene? \pause

\begin{itemize}
    \item \textbf{Duplicación de información}. \pause 34.122 lo estoy guardando en \texttt{mills}, pero después vuelvo a poner 34.112 en la conversión a kilómetros \pause
    \item ¿Cómo se podría arreglar?
\end{itemize}
\end{frame}

% PAG 2
\begin{frame}[fragile]{¿Solución?} 
\begin{lstlisting}[language=Python]
#ingreso las millas 
mills = 34.122

#hago la conversion a kilometros
km = mills * 1.6

#imprimo por pantalla el resultado
print(mills, "millas son", km, "kilometros")

#la salida queda 
> 34.122 millas son 54.595200000000006 kilometros
\end{lstlisting} \pause

\begin{itemize}
    \item \textbf{Duplicación de información corregida}\pause : 34.122 lo estoy guardando en \texttt{mills}, y después uso la  variable \texttt{mills} y no vuelvo a poner 34.112 en la conversión \pause
    \item Sin embargo, la salida no es del todo linda...
\end{itemize}
\end{frame}


% PAG 3
\begin{frame}[fragile]{¿Qué hace el siguiente programa en Python? (II)}
\begin{lstlisting}[language=Python]
#ingreso las millas 
mills = 34.122

#hago la conversion a kilometros
km = mills * 1.6

round(54.595200000000006, 2)

#imprimo por pantalla el resultado
print(mills, "millas son", km, "kilometros")

#la salida queda 
> 34.122 millas son 54.595200000000006 kilometros
\end{lstlisting} \pause

\begin{itemize}
    \item ¿Qué pasó con la salida? \pause ¿El programa es el mismo? \pause
    \item ¿Qué problema tiene? \pause El resultado de \texttt{round(54.595200000000006, 2)} \textbf{no se guarda en ningún lado}.\pause  ¿Por qué? \pause ¿Cómo podría hacerlo?\pause 
    \item ¿Qué otro problema hay? \pause Nuevamente hay duplicación de información: en vez de \texttt{54.595200000000006} podemos usar \texttt{km} que es la variable que guarda ese valor
\end{itemize}
\end{frame}


%PAG 4
\begin{frame}[fragile]{Guardo el resultado de redondear y reutilizo su valor en el programa}
\begin{lstlisting}[language=Python]
#ingreso las millas 
mills = 34.122

#hago la conversion a kilometros
km = mills * 1.6

km_redondo = round(km, 2)

#imprimo por pantalla el resultado
print(mills, "millas son", km, "kilometros")

#la salida queda 
> 34.122 millas son 54.595200000000006 kilometros
\end{lstlisting} \pause

\begin{itemize}
    \item ¿Qué hace ahora el programa? \pause En esencia, ¿es distinto que el programa anterior? \pause
    \item ¿Solucionó el problema identificado? \pause ¿Se utilizó en algun lado la variable \texttt{km\_redondo}? ¿Cómo podría utilizarlo? 
\end{itemize}
\end{frame}


%PAG 5
\begin{frame}[fragile]{Ahora sí?}

\begin{lstlisting}[language=Python]
#ingreso las millas 
mills = 34.122

#hago la conversion a kilometros
km = mills * 1.6

km_redondo = round(km, 2)

#imprimo por pantalla el resultado
print(mills, "millas son", km_redondo, "kilometros")

#la salida queda 
> 34.122 millas son 54.59 kilometros
\end{lstlisting} \pause

\begin{itemize}
    \item Este programa, en vez de mostrar (imprimir por pantalla) el valor de \texttt{km}, muestra el valor redondeado asignado a la variable \texttt{km\_redondeado}
\end{itemize}
\end{frame}


%PAG 6
\begin{frame}[fragile]{Utilizando Listas}
\begin{lstlisting}[language=Python]
#ingreso las millas 
mills = [34.122, 17.588, 3.187] 

# Hago la conversion a kilometros
km = [mills[0]*1.6, mills[1]*1.6, mills[2]*1.6]

# Imprimo por pantalla el resultado
print(mills[0],"millas son",km[0],"kilometros")  
print(mills[1],"millas son",km[1],"kilometros")  
print(mills[2],"millas son",km[2],"kilometros")
\end{lstlisting} \pause

La salida ahora queda:
\begin{lstlisting}
> 34.122 millas son 54.5952 kilometros
> 17.588 millas son 28.1408 kilometros
> 3.187 millas son 5.0992 kilometros
\end{lstlisting} \pause

\begin{itemize}
    \item ¿Cuál es la diferencia fundamental con los programas anteriores? \pause Que ahora trabajamos con listas de \texttt{floats} y no con un único valor \pause
    \item ¿Tiene información duplicada? \pause ¿Cómo se podría evitar?
\end{itemize}
\end{frame}

%PAG 7 
\begin{frame}[fragile]{Estructuras de control} \pause
\begin{itemize}
    \item Las estructuras de control nos permiten  modificar el flujo de ejecución del programa. \pause
    \item En otras palabras, quizas más intrincadas, permiten estructurar el proceso de ejecución a partir de ciertas \textbf{condiciones lógicas} que definimos. \pause
    \item Ejemplo: Condicional \texttt{if} \pause
\end{itemize}

\begin{lstlisting}[language=Python]
if CONDICION:
    PROG1 
\end{lstlisting} \pause

\begin{itemize}
    \item CONDICION es una expresión que arroja resultado veradero o falso
    \item PROG1 es un programa que hace algo
    \item PROG1 se ejecuta \textbf{si y solo si} CONDICION arroja valor verdadero \pause
    \item Atención con el bloque indentado!
\end{itemize}
\end{frame}

%PAG 8
\begin{frame}[fragile]{Ejemplo de if}
\begin{lstlisting}[language=Python]
if 1 > 5:
    print('1 es mayor que 5')
if 1 < 5:
    print('1 es menor que 5') 
\end{lstlisting} \pause

¿Cuál es su salida? \pause
\begin{lstlisting}[language=Python]
> '1 es menor que 5'
\end{lstlisting}
\end{frame}

% MEME IF
\begin{frame}{Con las estructuras de control se puede hacer cualquier cosa...} \pause
\begin{center}
\includegraphics[height=7cm, scale=0.5]{rick_ai.jpeg}
\end{center}
\end{frame}


%PAG 9
\begin{frame}[fragile]{Otro ejemplo de if}
\begin{lstlisting}[language=Python]
a = 10
b = [100, 1]

if b[0] // (a * 10) == b[1]:
    b[0] = b[0] - 1
    b[1] = b[1] * 5

print(a, b[0], b[1])
\end{lstlisting} \pause

¿Cuál es su salida? 3 minutos para pensarlo... \pause 
\begin{lstlisting}[language=Python]
> 10, 99, 5
\end{lstlisting}
\end{frame}

%PAG 10
\begin{frame}[fragile]{Condicional: if, elif, else} \pause

\begin{lstlisting}[language=Python]
if CONDICION1:
    PROG1
elif CONDICION2:
    PROG2
else:
    PROG3
\end{lstlisting} \pause

\begin{itemize}
    \item CONDICION1 y CONDICION2 son expresiones lógicas
    \item PROG1, PROG2 y PROG3  son programas 
    \item PROG1 se ejecuta \textbf{si y solo si} CONDICION1 arroja valor \texttt{TRUE}\pause
    \item De lo contrario, se evalua CONDICION2 y, \textbf{si es verdadera}, se ejecuta PROG2 \pause
    \item Si CONDICION1 y CONDICION2 arrojan valores \texttt{FALSE}, se ejecuta el PROGRAMA3 de la sentencia \texttt{else}
\end{itemize}
\end{frame}

%PAG 11
\begin{frame}[fragile]{Ciclos o Bucles}\pause
\begin{lstlisting}[language=Python]
while CONDICION:
    PROG1
\end{lstlisting} \pause
\begin{itemize}
    \item CONDICION es una expresión que arroja resultado \texttt{TRUE} o \texttt{FALSE}
    \item PROG1 es un programa que hace algo
    \item La ejecución de PROG1 se repite \textbf{mientras} CONDICION arroja valor \texttt{TRUE}
\end{itemize}
\end{frame}


%PAG 12
\begin{frame}[fragile]{Ejemplo de while}
\begin{lstlisting}[language=Python]
i = 0 #arranco la inicializacion en valor = 0
while i < 3:
    print(i) #imprimo por pantalla
    i = i+1 #muevo el indice una posicion
\end{lstlisting} \pause

¿Cuál es su salida? \pause 
\begin{lstlisting}[language=Python]
> 0 
> 1
> 2
\end{lstlisting}
\end{frame}

%PAG 13
\begin{frame}[fragile]{Ejemplo de if y while}
\begin{lstlisting}[language=Python]
i = 0
while i < 3:
    if i % 2 == 0:
        print(i, 'es par')
    else:
        print(i, 'es impar')
    i = i + 1

\end{lstlisting} \pause

¿Cuál es su salida? Algune se anima a soplarla? (piensen cuántas veces se ejecuta el programa del bloque indentado) \pause 
\begin{lstlisting}[language=Python]
> 0 es par
> 1 es impar
> 2 es par
\end{lstlisting}
\end{frame}

% MEME WHILE
\begin{frame}{Ojo con NO olvidarse la condición de fin del ciclo while!} \pause
\begin{center}
\includegraphics[height=7cm, scale=1]{infinite_loop.png}
\end{center}
\end{frame}


%PAG 14
\begin{frame}[fragile]{Ejemplo de condicional y ciclo un poquito más útil...}
\begin{lstlisting}[language=Python]
#ingreso las millas 
mills = [1.1, 33.4, 34.122, 17.588, 3.187, 50.] 
    
#guardo la longitud de la lista
longitud = len(mills)
    
#hago la conversion a kilometros
km = [0]*longitud #lista de ceros de longitud adecuada
    
i = 0 # variable para indicar la posicion en la lista
while i < longitud: #condicion
    km[i] = mills[i]*1.6 #conversion
    i = i +1 #avance de posicion
\end{lstlisting} \pause

\begin{itemize}
        \item ¿Qué hace \texttt{while i $<$ longitud}? \pause Nos indica que el bloque de código siguiente se va a ejecutar mientras el resultado de la condición lógica sea verdadera. \pause
        \item Es decir, en este caso, siempre que el índice \texttt{i} sea menor a la longitud de la lista (variable \texttt{longitud}) \pause 
        \item ¿Funciona este programa para la lista de cualquier longitud?
\end{itemize}
\end{frame}


% PAG final
\begin{frame}{A coedear se a dicho!} \pause
Tener presente que:  \pause
\begin{itemize}
	\item Cualquier \emph{valor que quiera ser reutilizado} (para un cálculo posterior, para una salida, etc.) \alert{debe ser almacenado previamente en una variable} \pause
		\item Siempre que sea posible \alert{no duplicar información} en el código. \pause
		\item  Y por último.... 
\end{itemize} 
\end{frame}

% MEME errores
\begin{frame}{No tenerle miedo a los errores!}
\begin{center}
\includegraphics[height=7cm, scale=0.5]{meme_error.jpg}
\end{center}
\end{frame}





\end{document}