%\documentclass[9pt]{beamer}
\documentclass[handout,9pt]{beamer}

\mode<presentation>
{
%  \usetheme{Warsaw}
  % or ...

%  \setbeamercovered{transparent}
  % or whatever (possibly just delete it)
}


\setbeamertemplate{navigation symbols}{}

\usepackage[utf8]{inputenc}
\usepackage[spanish, es-tabla, es-nodecimaldot]{babel}

\usepackage{listings}
\usepackage{tikzsymbols}

%\usepackage{times}
%\usepackage[T1]{fontenc}
% Or whatever. Note that the encoding and the font should match. If T1
% does not look nice, try deleting the line with the fontenc.


\title[Intro Prog] % (optional, use only with long paper titles)
{Introducción a la programación}

\subtitle
{Usando Python}

\author[SP]
{G. Sebastián Pedersen}%\inst{1}}
% - Give the names in the same order as the appear in the paper.
% - Use the \inst{?} command only if the authors have different
%   affiliation.

\institute[UNGS] % (optional, but mostly needed)
{
%  \inst{1}%
  Instituto de Industria\\
  Universidad Nacional de General Sarmiento
%  \and
%  \inst{2}%
%  Department of Theoretical Philosophy\\
%  University of Elsewhere}
% - Use the \inst command only if there are several affiliations.
% - Keep it simple, no one is interested in your street address.
}

\date[] % (optional, should be abbreviation of conference name)
{Matemática para Economistas III, 1er. cuat. 2019\\ 
\vspace{.5cm}
\url{https://sebasped.github.io/pythonungs/}
}



% If you have a file called "university-logo-filename.xxx", where xxx
% is a graphic format that can be processed by latex or pdflatex,
% resp., then you can add a logo as follows:

% \pgfdeclareimage[height=0.5cm]{university-logo}{university-logo-filename}
% \logo{\pgfuseimage{university-logo}}



% Delete this, if you do not want the table of contents to pop up at
% the beginning of each subsection:
%\AtBeginSubsection[]
%{
%  \begin{frame}<beamer>{Outline}
%    \tableofcontents[currentsection,currentsubsection]
%  \end{frame}
%}


% If you wish to uncover everything in a step-wise fashion, uncomment
% the following command: 

%\beamerdefaultoverlayspecification{<+->}


\begin{document}

\begin{frame}
  \titlepage
\end{frame}



\begin{frame}[fragile]{Resumen de conceptos importantes}
\begin{itemize}
	\item Variables: \pause las utilizamos para guardar valores y reutilizarlos posteriormente (en una cuenta, imprimirlos, etc.) \pause
	\item Ciclos \verb|while| y \verb|for|: \pause se utilizan para hacer muchas veces lo mismo cuando a priori no sabemos la cantidad total de veces que lo tenemos que hacer.\pause
	\item Condicional \verb|if|: \pause sirve para separa casos con comportamientos distintos.\pause
	\item Funciones.\pause
	\item Duplicación de información.\pause
	\item Particionar y encapsular el problema.
\end{itemize}
\end{frame}



\begin{frame}[fragile]{¿Cómo se usaba una función?}
% \includegraphics[width=2cm]{th.jpg}

Primero la defino:
\normalsize{\begin{verbatim}
def nombre_func(x,a):
    ...
    cuentas o lo que haga la función con la entrada
    ...
    
    return lo_que_devuelvo
\end{verbatim}}\pause
¿Cuál es la entrada y cuál es la salida?\pause
\vspace{0.5cm}

Y luego la utilizo:
\normalsize{\begin{verbatim}
salida = nombre_func(3,8)
print(salida)
\end{verbatim}}\pause
En la utilización, ¿cuánto valen la entrada y la salida?

\end{frame}



\begin{frame}[fragile]{Resolviendo ecuaciones}
% \includegraphics[width=2cm]{th.jpg}
Supongamos que quiero resolver la siguiente ecuación:
$$ -3x=2^x$$
O lo que es lo mismo
$$ -3x-2^x=0$$
¿Cómo hago? \pause
Analíticamente no se puede, y entonces cobra sentido resolverlas numéricamente con la compu.\pause

Existe varios métodos numéricos para dar una solución aproximada. Por ejemplo Newton-Raphson.
\end{frame}

\begin{frame}[fragile]{Resolviendo ecuaciones}
Este programita resuelve la ecuación $ -3x-2^x=0$ utilizando Newton-Raphson:
\normalsize{\begin{verbatim}
from scipy import optimize

a=-3
b=2

def f(x):
    func = a*x - b**x 
    return func

# la entrada para optimize es la función y un valor inicial.
# resuelve por N-R la ecuación f=0
raiz = optimize.newton(f, 1.5)
print(raiz)

comprobar = a*raiz-b**raiz
print(comprobar)
\end{verbatim}}\pause
La salida de \verb|print(raiz)| da: -0.27540593648582107 \pause

La salida de \verb|print(comprobar)| da: -2.220446049250313e-16

\end{frame}




\begin{frame}[fragile]{Calculando Yield to Maturity (YTM)}
Si ahora quiero calcular la YTM me voy a encontrar con una ecuación de este estilo
$$ p = \dfrac{c}{(1+y/n)^{1}} + \dfrac{c}{(1+y/n)^{2}} + \dots + \dfrac{100+c}{(1+y/n)^{nT}}$$
Donde:
\begin{itemize}
	\item $p$ es el precio del bono
	\item $c$ es el cupón (lo que paga en cada período en \%)
	\item $T$ son los años hasta el vencimiento.
	\item $n$ es la frecuencia de pago por año.
	\item $y$ es la YTM (lo que quiero calcular).
\end{itemize}
Entonces estamos en una situación similar a la anterior: la incógnita $y$ no se puede despejar. \pause

O lo que es lo mismo, calcular la raíz de:
$$ \dfrac{c}{(1+y/n)^{1}} + \dfrac{c}{(1+y/n)^{2}} + \dots + \dfrac{100+c}{(1+y/n)^{nT}} - p = 0$$

Por lo tanto tiene sentido encontrar una solución aproximada por Newton-Raphson utilizando la compu.
\end{frame}


\begin{frame}[fragile]{Calculando Yield to Maturity (YTM)}
Resumiendo, tenemos una $f(y)=0$:
$$ \underbrace{\dfrac{c}{(1+y/n)^{1}} + \dfrac{c}{(1+y/n)^{2}} + \dots + \dfrac{100+c}{(1+y/n)^{nT}} - p}_{f(y)} = 0$$

Donde:
\begin{itemize}
	\item $p$, $c$, $T$ y $n$ son valores ya dados.
	\item Queremos encontrar el valor de $y$ que satisface la ecuación.
\end{itemize}
\end{frame}


\begin{frame}{¡Twist and Code! Ejercicios:}
%\begin{block}{Ejercicios:}
	\begin{enumerate}
		\item Hacer una función que reciba como entrada un número $n\in\mathbb{N}$, y devuelva como salida la suma de los números entre $1$ y $n$.
		
		Testear la función para varios casos.
		
		\item Hacer una función que reciba como entrada dos números $n\in\mathbb{N}$ y $k\in\mathbb{N}$, y devuelva como salida la siguiente suma: $k^1 + k^2 + \dots +k^{n-1} + k^n$.

%Testear la función para varios casos.
		\item Hacer una función que reciba como entrada dos números $n\in\mathbb{N}$ y $k\in\mathbb{N}$, y devuelva como salida la siguiente suma: $\frac{1}{k^1} + \frac{1}{k^2} + \dots +\frac{1}{k^{n-1}} + \frac{1}{k^n}$.

		\item Hacer una función que reciba como entrada tres números $n\in\mathbb{N}$, $k\in\mathbb{N}$ y $M\in\mathbb{R}$, y devuelva como salida la  siguiente suma: $\frac{1}{k^1} + \frac{1}{k^2} + \dots +\frac{1}{k^{n-1}} + \frac{1}{k^n} + M$.

		\item Hacer un programa que resuelva por Newton-Raphson la YTM.
		\begin{itemize}
			\item Usar la fórmula vista en estas diapos, o cualquier otra fórmula donde la YTM no se pueda despejar.
			\item Se puede suponer que los parámetros además de la YTM (el precio, el cupón, años hasta el vencimiento, etc.) ya están dados (se pueden fijar antes de definir a la función necesaria para N-R. Ver la 2da. diapo de ``Resolviendo ecuaciones'').
			\item El programa tiene que funcionar para \emph{cualquier} valor de los parámetros.
		\end{itemize}
		
%		Suponer que el precio $p$, el cupón $c$, los años hasta el vencimiento $T$ y la frecuencia de pago por año $n$ son parámetros que ya están dados 
		
%		\item Utilizar el programa para calcular algunas YTMs.
	\end{enumerate}
%\end{block}

\end{frame}





\end{document}


