\documentclass{beamer}
%\documentclass[handout]{beamer}

\mode<presentation>
{
%  \usetheme{Warsaw}
  % or ...

%  \setbeamercovered{transparent}
  % or whatever (possibly just delete it)
}


\setbeamertemplate{navigation symbols}{}

\usepackage[utf8]{inputenc}
\usepackage[spanish, es-tabla, es-nodecimaldot]{babel}

\usepackage{listings}
\usepackage{tikzsymbols}

%\usepackage{times}
%\usepackage[T1]{fontenc}
% Or whatever. Note that the encoding and the font should match. If T1
% does not look nice, try deleting the line with the fontenc.


\title[Intro Prog] % (optional, use only with long paper titles)
{Introducción a la programación}

\subtitle
{Usando Python}

\author[SP]
{G. Sebastián Pedersen}%\inst{1}}
% - Give the names in the same order as the appear in the paper.
% - Use the \inst{?} command only if the authors have different
%   affiliation.

\institute[UNGS] % (optional, but mostly needed)
{
%  \inst{1}%
  Instituto de Industria\\
  Universidad Nacional de General Sarmiento
%  \and
%  \inst{2}%
%  Department of Theoretical Philosophy\\
%  University of Elsewhere}
% - Use the \inst command only if there are several affiliations.
% - Keep it simple, no one is interested in your street address.
}

\date[] % (optional, should be abbreviation of conference name)
{Matemática para Economistas III, 1er. cuat. 2019\\ 
\vspace{.5cm}
\url{https://sebasped.github.io/pythonungs/}
}



% If you have a file called "university-logo-filename.xxx", where xxx
% is a graphic format that can be processed by latex or pdflatex,
% resp., then you can add a logo as follows:

% \pgfdeclareimage[height=0.5cm]{university-logo}{university-logo-filename}
% \logo{\pgfuseimage{university-logo}}



% Delete this, if you do not want the table of contents to pop up at
% the beginning of each subsection:
%\AtBeginSubsection[]
%{
%  \begin{frame}<beamer>{Outline}
%    \tableofcontents[currentsection,currentsubsection]
%  \end{frame}
%}


% If you wish to uncover everything in a step-wise fashion, uncomment
% the following command: 

%\beamerdefaultoverlayspecification{<+->}


\begin{document}

\begin{frame}
  \titlepage
\end{frame}



\begin{frame}{Máximas de la programación que vimos hasta ahora}
\begin{itemize}
	\item \alert{Reutilización de valores:} \pause cualquier \emph{valor que quiera ser reutilizado} (para un cálculo posterior, para una salida, etc.) \emph{debe ser almacenado previamente en una variable}.\pause
	\item \alert{Duplicación de información:} \pause siempre que sea posible \emph{no duplicar información} en el código.\pause
		\begin{itemize}
			\item Guardar el valor que se repite en una variable, y utilizar la variable en lugar del valor.\pause
			\item Con las líneas de código que se repiten crear una función, y utilizarla en lugar de andar repitiendo el código.
		\end{itemize}\pause
	\item \alert{Particionar y encapsular el problema:} siempre que sea posible \emph{particionar} el problema y \emph{encapsularlo} en problemas menos complejos.\pause
		\begin{itemize}
			\item Mediante funciones.\pause
			\item Mediante scripts.
		\end{itemize}
\end{itemize}
\end{frame}



\begin{frame}[allowframebreaks,fragile]{¡Nuestro primer programa para resolver ODEs!}
 \includegraphics[width=2cm]{th.jpg}
%¿Cuál es la diferencia con el anterior programa?
\scriptsize{\begin{verbatim}
	#para resolver ecuaciones diferenciales
	from sympy import *
	# para generar una lista de valores equiespaciados
	from numpy import linspace
	
	# defino los nombres de la variable y la función
	t = symbols('t') # la variable independiente
	y = symbols('y', cls=Function) # la función
	
	# defino la ecuación diferencial
	# Eq es la función de Python para definidir ecuaciones diferenciales. La coma es como el igual.
	# y(t).diff(t) es la derivada primera
	# y(t).diff(t,t) es la derivada segunda
	# defino  y'' + 6y' + 9y = 45  (cambiar a gusto)
	ecDif = Eq( y(t).diff(t,t) + 6*y(t).diff(t) + 9*y(t), 45 )
	
	# defino condiciones iniciales y(0)=1 e y'(0)=2  (cambiar a gusto)
	condInic = {y(0): 1, y(t).diff(t).subs(t, 0): 2}
	
	# resuelvo la ecuación diferencial
	# dsolve es la función de Python para calcular la ec. dif.
	# ecDif es la ecuación a calcular
	# y(t) son la variable y función a resolver
	# ics son las condiciones iniciales
	sol = dsolve( ecDif, y(t), ics=condInic )
	
	#Muestro la ecuación y el resultado
	print 'La ecuación diferencial:'
	print ecDif
	print condInic
	print "\n" #agrega un renglón en blanco
	print 'La solución:'
	print sol
	print "\n" #agrega un renglón en blanco
	\end{verbatim}}

\includegraphics[height=2.4cm]{solEcDif.png}
%\begin{itemize}
%	\item La función \verb|convertir| ahora toma dos parámetros de entrada.\pause
%	\item La ventaja de esto es que ahora la función \verb|convertir| funciona \alert{para cualquier lista y para cualquier factor de conversión}.
	%	\item Ventajas de esto:\pause
	%	\begin{itemize}
	%		\item \footnotesize{Particionamos} el problema es dos problemas de menor complejidad.\pause
	%		\item Los dos problemas pueden sen resueltos por diferentes personas y al mismo tiempo.
	%	\end{itemize}
	%	\item ¿Cómo se podría arreglar?
%\end{itemize}

\end{frame}




\begin{frame}[fragile]{Pequeña mejora al mostrar la EDO y la solución}
\scriptsize{\begin{verbatim}	
	#Muestro la ecuación y el resultado de forma más legible
	
	#para resolver ecuaciones diferenciales
	from sympy import *
	#para imprimir fórmulas matemáticas
	init_printing() 
	from IPython.display import display 
	
	#muestro la ode y la solución
	print 'La ecuación diferencial:'
	display(ecDif)
	display(condInic)
	print "\n" #agrega un renglón en blanco
	print 'La solución:'
	display(sol)
\end{verbatim}}
\includegraphics[height=3cm]{solEcDifMasCopada.png}
%\begin{itemize}
%	\item La función \verb|convertir| ahora toma dos parámetros de entrada.\pause
%	\item La ventaja de esto es que ahora la función \verb|convertir| funciona \alert{para cualquier lista y para cualquier factor de conversión}.
%	\item Ventajas de esto:\pause
%	\begin{itemize}
%		\item \footnotesize{Particionamos} el problema es dos problemas de menor complejidad.\pause
%		\item Los dos problemas pueden sen resueltos por diferentes personas y al mismo tiempo.
%	\end{itemize}
%	\item ¿Cómo se podría arreglar?
%\end{itemize}

\end{frame}



\begin{frame}[fragile]{También graficando la solución}
\scriptsize{\begin{verbatim}	
	# para graficar
	from matplotlib import pyplot as plt
	
	f = sol.args[1] # me quedo solamente con la fórmula de la solución
	lam = lambdify(t, f, modules=['numpy']) #hago la fórmula evaluable
	t_vals = linspace(0, 20, 100) #valores de t a usar para evaluar: inicio, fin, paso.
	y_vals = lam(t_vals) #evalúo la fórmula en los valores de t
	
	#grafico la solución
	plt.plot(t_vals, y_vals) #hago el gráfico
	plt.grid() #que me ponga un grillado en el gráfico
	plt.xlabel('Valores de t') #leyende del eje horizontal
	plt.ylabel('Valores de y') #leyenda del eje vertical
	plt.show() #muestro el gráfico
	\end{verbatim}}
\includegraphics[height=3.5cm]{solEcDifGrafico.png}
\end{frame}


%\begin{frame}[fragile]{¡Nuestro primer programa para resolver ODEs!}
%%¿Cuál es la diferencia con el anterior programa?
%\footnotesize{\begin{verbatim}
%#para resolver ecuaciones diferenciales
%from sympy import *
%#para imprimir fórmulas matemáticas
%init_printing() 
%from IPython.display import display 
%
%# para generar una lista de valores equiespaciados
%from numpy import linspace
%
%# para graficara
%#import matplotlib.pyplot as plt
%from matplotlib import pyplot as plt
%
%
%# defino los nombres de la variable y la función
%t = symbols('t') # la variable independiente
%y = symbols('y', cls=Function) # la función
%
%
%# defino la ecuación diferencial
%# Eq es la función de Python para definidir ecuaciones diferenciales. La coma es como el igual.
%# y(t).diff(t) es la derivada primera
%# y(t).diff(t,t) es la derivada segunda
%# defino  y'' + 6y' + 9y = 45  (cambiar a gusto)
%ecDif = Eq( y(t).diff(t,t) + 6*y(t).diff(t) + 9*y(t), 45 )
%#ecDif = Eq( y(t).diff(t) - 3*y(t), 12 ) #ejemplo orden 1:  y' -3y = 12
%
%
%# defino condiciones iniciales y(0)=1 e y'(0)=2  (cambiar a gusto)
%condInic = {y(0): 1, y(t).diff(t).subs(t, 0): 2}
%#condInic = {y(0): 4} #ejemplo orden 1:  y(0)=4
%
%# resuelvo la ecuación diferencial
%# dsolve es la función de Python para calcular la ec. dif.
%# ecDif es la ecuación a calcular
%# y(t) son la variable y función a resolver
%# ics son las condiciones iniciales
%sol = dsolve( ecDif, y(t), ics=condInic )
%
%
%# imprimo la ecuación diferencial
%print 'La ecuación diferencial:'
%display(ecDif)
%display(condInic)
%print "\n" #agrega un renglón en blanco
%
%# imprimo la solución
%print 'La solución:'
%display(sol)
%
%
%# si la anterior forma de mostrar la ec. dif. y la solución no funciona
%# se pueden imprimir más rudimentariamente así
%# (descomentar este y comentar el anterior)
%#print 'La ecuación diferencial:'
%#print ecDif
%#print condInic
%#print "\n" #agrega un renglón en blanco
%#print 'La solución:'
%#print sol
%#print "\n" #agrega un renglón en blanco
%
%
%#grafico la solución
%f = sol.args[1] # me quedo solamente con la fórmula de la solución
%lam = lambdify(t, f, modules=['numpy']) #hago la fórmula evaluable
%t_vals = linspace(0, 20, 100) #valores de t a usar para evaluar: inicio, fin, paso.
%y_vals = lam(t_vals) #evalúo la fórmula en los valores de t
%
%
%plt.plot(t_vals, y_vals) #hago el gráfico
%plt.grid() #que me ponga un grillado en el gráfico
%plt.xlabel('Valores de t')
%plt.ylabel('Valores de y')
%plt.show() #muestro el gráfico
%\end{verbatim}}\pause
%\begin{itemize}
%	\item La función \verb|convertir| ahora toma dos parámetros de entrada.\pause
%	\item La ventaja de esto es que ahora la función \verb|convertir| funciona \alert{para cualquier lista y para cualquier factor de conversión}.
%%	\item Ventajas de esto:\pause
%%	\begin{itemize}
%%		\item \footnotesize{Particionamos} el problema es dos problemas de menor complejidad.\pause
%%		\item Los dos problemas pueden sen resueltos por diferentes personas y al mismo tiempo.
%%	\end{itemize}
%	%	\item ¿Cómo se podría arreglar?
%\end{itemize}
%
%\end{frame}




\begin{frame}{To code or not to code...}
\alert{Tener presente las máximas de la programación} (ver diapo ``Máximas de la programación que vimos hasta ahora'').
\begin{block}{Ejercicios:}
	\begin{enumerate}
		\item \emph{Utilizar} nuestro primer programa que resuelve EDOs para resolver algunas ecuaciones diferenciales de la práctica.
		\item \emph{Modificar} nuestro primer programa que resuelve EDOs para muestre la ecuación y la solución de forma más legible (ver diapo ``Pequeña mejora al mostrar la EDO y la solución'').
		\item \emph{Modificar} el programa del punto 2. para que también grafique la solución de la EDO (ver diapo ``También graficando la solución'').
		\item \alert{¡Bonus Hacker! \dCooley:} \emph{Modificar} el primer programa que resuelve EDOs para que resuelva sistemas con 2 ecuaciones diferenciales (aunque sea algún caso básico). 
	\end{enumerate}
\end{block}

\end{frame}






%%
%%% All of the following is optional and typically not needed. 
%%\appendix
%%\section<presentation>*{\appendixname}
%%\subsection<presentation>*{For Further Reading}
%%
%%\begin{frame}[allowframebreaks]
%%  \frametitle<presentation>{For Further Reading}
%%    
%%  \begin{thebibliography}{10}
%%    
%%  \beamertemplatebookbibitems
%%  % Start with overview books.
%%
%%  \bibitem{Author1990}
%%    A.~Author.
%%    \newblock {\em Handbook of Everything}.
%%    \newblock Some Press, 1990.
%% 
%%    
%%  \beamertemplatearticlebibitems
%%  % Followed by interesting articles. Keep the list short. 
%%
%%  \bibitem{Someone2000}
%%    S.~Someone.
%%    \newblock On this and that.
%%    \newblock {\em Journal of This and That}, 2(1):50--100,
%%    2000.
%%  \end{thebibliography}
%%\end{frame}

\end{document}


